\chapter{囚禁在光学晶格中的超冷玻色原子,其基态相位可被玻色-哈伯德模型捕获}

\section{第一节}
囚禁在光学晶格中的超冷玻色原子,其基态相位可被玻色-哈伯德模型捕获。

\makeatletter\def\@captype{table}\makeatother
{\wuhao[1.5]
\begin{table*}
\centering \caption{不同序参量下量子相的分类}
\begin{tabular}{ccccccccc}
\hline\hline 量子相 &$|\Delta^{\sigma}_{i}|$ &
$n^{\sigma}_{i}$ & $\Delta n^{\sigma}$ & $N_{i}$
\\
\hline 
 MI &$=0$   &整数 (UN)& $=0$ & 整数 (UN) \\
 DW &$=0$   &整数 (CB)& $\neq0$ (整数) & 整数 (UN) \\
cMI &$=0$  &非整数 (UN)  & $=0$ & 整数(UN)  \\
cDW &$=0$  &非整数 (CB) & $\neq0$ (non-integer) & 整数(CB) \\
 SF &$\neq0$ & 实数 & $=0$ &实数 \\
DCSS  &$\neq0$ &实数(CB)& $\neq0$ (实数) & 实数(UN)\\
DSSS  &$\neq0$ &实数(ST)& $\neq0$ (实数) & 实数(UN)\\
 LSS  &$\neq0$ &实数(ST)& $\neq0$ (实数) & 实数(LT)\\
 NMSS  &$\neq0$ &实数(LT (spin-$\uparrow$) and UN (spin-$\downarrow$))& $\neq0$ (实数) & 实数(LT)\\
 \hline 
\end{tabular}
\begin{tablenotes}
       \footnotesize
       \item[1]{\(|\Delta_i^{\sigma}|\) 表示超流序的振幅,\(n_i^{\sigma}\) 和 \(\Delta n^{\sigma}\) 分别表示自旋-\(\sigma\) 玻色子的占据数和相对占据数,\(N_i\) 表示格点总的占据数,“UN” 表示均匀分布,“ST” 表示条纹结构,“LT” 表示晶格结构,“CB” 表示棋盘结构。“MI” 表示莫特绝缘相,“DW” 表示密度波相,“cMI” 表示关联莫特绝缘相,“cDW” 表示关联密度波相,“SF” 表示超流相,“DCSS” 表示双棋盘超固相,“DSSS” 表示双条纹超固相,“LSS” 表示晶格超固相,“NMSS” 表示非整数填充莫特超固相。}
     \end{tablenotes}
     \label{table:4-1}
\end{table*}
}


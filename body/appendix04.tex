\chapter{资产之间的相关性在组合构建中起到关键作用,合理的资产配置可以在不降低预期收益的情况下减少整体风险}
\label{appendix:BH}

\section*{相关理论基础}

\subsection*{(一)现代投资组合理论}
现代投资组合理论(Modern Portfolio Theory, MPT)由Harry Markowitz于1952年提出,其核心思想是通过分散投资来降低非系统性风险。该理论强调投资者应关注投资组合的整体风险-收益特征,而非单个资产的独立表现。根据MPT,资产之间的相关性在组合构建中起到关键作用,合理的资产配置可以在不降低预期收益的情况下减少整体风险。

该理论的主要数学表述包括:

期望收益:投资组合的预期收益是各资产预期收益的加权平均,公式为
$$E(R_p) = \sum_{i=1}^n w_iE(R_i),$$
其中 \( w_i \) 为资产 \( i \) 的权重,\( E(R_i) \) 为资产 \( i \) 的期望收益。

组合方差:衡量组合风险的方差由资产间的协方差矩阵决定,表达式为
$$\sigma_p^2 = \sum_{i=1}^n\sum_{j=1}^n w_iw_j\sigma_{ij},$$
其中 \( \sigma_{ij} \) 是资产 \( i \) 和 \( j \) 的协方差。

有效前沿:在风险-收益平面上,有效前沿代表在给定风险水平下能够提供最高预期收益的投资组合集合,或在给定收益目标下风险最小的组合。投资者可根据自身风险偏好选择前沿上的最优组合。

\subsection*{(二)夏普比率}
夏普比率(Sharpe Ratio)由William Sharpe于1966年提出,是衡量投资组合风险调整后收益的重要指标。其定义为组合超额收益(超过无风险利率的部分)与收益波动性的比值,公式为
$$\text{Sharpe Ratio} = \frac{E(R_p) - R_f}{\sigma_p},$$
其中 \( E(R_p) \) 为组合预期收益,\( R_f \) 为无风险利率(如国债收益率),\( \sigma_p \) 为组合收益的标准差(代表总风险)。夏普比率越高,表明单位风险所获得的超额收益越高,投资效率更优。该指标广泛用于比较不同组合或基金的绩效表现。

\subsection*{(三)组合优化方法}
在实际应用中,投资组合优化通常通过数学模型实现,常见方法包括:

均值-方差优化:基于Markowitz框架,通过二次规划求解有效前沿上的组合权重。目标函数为最小化组合方差(风险)或最大化预期收益,需输入资产的预期收益、方差及协方差矩阵。

最大夏普比率组合:寻找使夏普比率最大化的切线组合(Tangency Portfolio)。该组合位于有效前沿与资本配置线(CAL)的切点,需通过非线性优化方法求解。

约束优化:在现实投资中,需考虑交易成本、流动性、法规限制等约束条件。例如,禁止卖空(权重非负)、设置行业或资产类别上限等。此类问题通常采用带约束的数值优化算法(如序列二次规划)处理。

这些方法为构建科学化、定量化的投资策略提供了理论基础,但需注意对输入参数(如预期收益和协方差)的估计误差可能显著影响优化结果。